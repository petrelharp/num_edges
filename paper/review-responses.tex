\begin{minipage}[b]{2.5in}
  Resubmission Cover Letter \\
  {\it Genetics}
\end{minipage}
\hfill
\begin{minipage}[b]{2.5in}
    Halley Fritze  \\
  \today
\end{minipage}

\vskip 2em

\noindent
{\bf To the Editor(s) -- }

\vskip 1em

We are writing to submit a revised version of our manuscript,
GENETICS-2025-307917 
titled
`` A forest is more than its trees: haplotypes and ancestral recombination graphs''.
We appreciate the thoughtful comments by the reviewers.
We've done our best to address all the comments,
as detailed in our Response to Reviewers.

In addition to a Response to Reviewers (in which page/line numbers refer to the revised manuscript file),
we also provide a pdf with differences to the previously submitted version highlighted.

The main changes we've made are:
(1) additional detail and description to guide the reader through some of the sections;
(2) a fix to a bug in the algorithm
(thanks, Reviewer 3; note this did not affect the code, just the paper);
and (3) more description of how extending haplotypes affects efficiency on inferred ARGs.

\vskip 2em

\noindent \hspace{4em}
\begin{minipage}{3in}
\noindent
{\bf Sincerely,}

\vskip 2em

{\bf
   Halley Fritze, Nathaniel Pope, Jerome Kelleher, and Peter Ralph
}\\
\end{minipage}

\vskip 4em

\pagebreak

%%%%%%%%%%%%%%
\reviewersection{1}

\begin{quote}
In this paper Fritze and colleagues take a haplotype-centric view of the ancestral recombination graph (ARG) and contrast this view with the more common simplification of the ARG as a sequence of marginal trees. In particular, internal nodes in the ARG represent ancestral haplotypes, but in the process of converting an ARG to a succinct tree sequence, some of these internal nodes are deleted (essentially if one of the lineages that descend from the node recombines and coalesces elsewhere, then that node would become unary and is not represented in the succinct tree sequence). As such, a given node in a succinct tree sequence may not span as many marginal trees as it ``should'', and this paper develops an algorithm to try to infer where these unary nodes should be added, which the authors refer to as extending haplotypes. To assess how well their algorithm can recover trees, the authors develop novel dissimilarity measures on ARGs, and use these on simulated true and inferred ARGs. Overall, the extension method works quite well on true ARGs, but less well on inferred ARGs. I think that this paper is an important contribution to the ARG literature -- this haplotype perspective is important and generally overlooked, and measures for assessing how well methods can recover ARGs in terms of this perspective will be useful for future method development. Overall the paper is very nice, well-written, interesting, and clear, but I do have a few comments for the authors in the hopes that they're useful. They're all minor, and the authors should feel free to just use what they find helpful.
\end{quote}

\begin{point}{}
 I really enjoyed the title of the paper!
\end{point}

\reply{
Thank you!
}

\begin{point}{\revref}
 I am a bit confused by the definition of weighted Robinson-Foulds distance at \revref.% on the bottom of page 6. 
I believe that this is more like a Robinson-Foulds \emph{similarity}, as I believe that as defined it should be maximized when the two ARGs are the same, and smaller for more dissimilar ARGs.
\end{point}

\reply{
    Oh dear, that was an embarrassing error
    (that happily doesn't affect anything else in the paper).
    This has been corrected.
}

\begin{point}{}
 Is the dissimilarity measure defined in equation \eqref{eqn:wrmse} used anywhere?
\end{point}

\reply{No, the weighted mean square error defined in \eqref{eqn:wrmse} is not shown for our later experiments shown in \figref{fig:dissimilarity}, however it is always computed when using our implementation is \texttt{tscompare};
we now explicitly say this \revref.
}

\begin{point}{}
 What is ``dissimilarity'' as mentioned in the caption of \figref{fig:conceptual_discrepancy}?
\end{point}

\reply{``dissimilarity'' was our original language to describe the non-matched spans, and we missed this use of the old name. We have now changed the caption in \figref{fig:conceptual_discrepancy} to reflect that change.
}

\begin{point}{\revref}
 I found the discussion around the metric version to be hard to follow and worth expanding.
For example, I believe that the edges of the ARG are what's getting embedded into
$[0, L) \times N \times N$, and the symmetric differences are taken in this space in the obvious way (i.e., without this mapping its unclear what the relative difference of two ARGs is).
There's also a mention of ``dissimilarity'' in this paragraph. I assume that that's the ``symmetrized version of non-matched span'' that was just defined, but it would be good to make that clear. 
\end{point}

\reply{
    Thanks again to the reviewer for carefully thinking about what we wrote,
    which on further examination has turned out to be wrong in several ways (apologies!).
    Please see the updated version that we believe to be correct
    and thus hopefully less confusing. \revref
}

\begin{point}{}
 The performance on inferred ARGs makes clear that this is a difficult problem, and that this haplotype view might be useful conceptually and for methods development, but that as of now we should probably not take these ancestral haplotypes at face value in inferred ARGs. It would be helpful if the authors could provide some intuition for what's going wrong in this case. Presumably the topologies are not inferred well, and many nodes can be extended, which often results in some true and false positives? Another use case I could see would be using this approach to simplify inferred ARGs to make various downstream computations faster (a la \figref{fig:speed_and_edges}). As such, I would be curious to see how the extension algorithm affects the number of edges in inferred ARGs.
\end{point}

\reply{
    We agree on all counts -- in particular,
    it would be helpful to dig into why haplotypes are being inferred wrongly;
    however, this more detailed investigation is beyond the scope of this paper
    (since it's more an investigation into the inference methods).
    More generally: if an inferred ARG is incorrect for certain nodes, our method of extend haplotypes exacerbates this issue as extending an incorrectly inferred node extends this incorrect node span even further; we've added a note to this effect \revref.
    We've also added \figref{fig:inferred_runtime} that shows numbers of edges and runtime
    for the inferred ARGs used in our tests,
    and additional discussion of this point \llname{inferred_runtime}.
    The results are interesting, but detailed investigation is outside the scope of this paper --
    here we are introducing concepts and tools (and, there's plenty to digest);
    while further investigation could depend strongly on the inference method used.
}

\begin{point}{}
 This is obviously a matter of taste, but I would have found the paper a bit easier to follow if some of the points discussed in the discussion were brought up earlier in the paper. For example, the ``ignorance and omission in the ARG'' section \llname{ignorance}
answered questions I had when reading the bits around \figref{fig:conceptual} -- upon initially reading the bit around \figref{fig:conceptual}
I was confused by the interpretation that the lack of a node implies that that individual is not represented in other parts of the ARG (as any point along any branch in an ARG represents \emph{some} individual ancestral to sample), and as a result I found the presented interpretation somewhat unnatural.
This was then eventually clarified (very eloquently) in the discussion.
Similarly, I would have appreciated the connection to parsimony earlier.
It was not completely obvious to me that recovering the unary nodes from the succinct tree sequence might not be possible in all cases, and as such that the algorithm is an attempt at inference, and furthermore, that it's optimizing some reasonable objective. Again, this point was addressed very well, but I had to wait until the discussion.
\end{point}

\reply{
    This is a good suggestion;
    rather than break up the flow (for less sophisticated readers)
    we've inserted a pointer to these parts of the discussion, at \revref.
}




%%%%%%%%%%%%%%
\reviewersection{2}

\begin{quote}
The authors present interesting results in a clear and well-written manuscript, providing a deeper understanding of Ancestral Recombination Graphs and developing improved inference methods. The key idea is to take into account the additional information provided by haplotype structure which has rarely been considered before. On the way, the authors also develop a haplotype-aware measure of (dis)agreement between ARGs. In particular, the achieved goal is to infer information about the intermediate (unary) nodes in the local trees and to show how accurate the inference is.
\end{quote}

\begin{point}{\revref} % p.2 l.53
 ``Notation and terminology'': it would be good to clarify at this point that parents and children in the graph correspond to ancestors and descendants in the population.
\end{point}

\reply{Good observation. We have made this change. \revref
}

\begin{point}{\revref} % p.4 l.10
 ``in practice we do repeated passes in both directions until no changes can be made'': to someone who does not have hands-on experience with the implementation of this algorithm, the reason why the algorithm needs to do several passes in both directions might not be very clear. If at each tree transition all pair of mergeable paths are merged, it would be good to explain why, after one pass in each direction, additional mergeable paths are found. Perhaps explaining when/why additional mergeable paths are found/created could also give further insight on how many passes will be needed?
\end{point}

\reply{Noted. We have written some follow up for that sentence at your suggestion. The creation of new mergable paths after a number of passes is something we thought about quite a bit, however could not come up with a good theoretical conclusion as to if there was a number of required passes before termination. We instead provide some empirical evidence to the number of required passes later in the section and \tabref{tab:edge-counts}. \revref
}

\begin{point}{\revref} % p.8 l.25
 ``we simulate ARGs containing full haplotypes using msprime'': since msprime can simulate ARGs under various models, please specify here which model is used (SMC, SMC'... ?).
\end{point}

\reply{
Good suggestion; we've done this (we used ``hudson''). \revref
}

\begin{point}{}
In general, it comes natural to wonder whether the algorithms can be corrected to avoid incorrectly added spans
(e.g. \revref). % p.10 l. 5).
It would be worth to mention whether an attempt of correction has been made and to explain the difficulties/limitations involved.
\end{point}

\reply{
    Good suggestion; we've added a note about this. \llname{bad_spans}
}

\begin{point}{\revref} % p.12 l.9
 ``Data availability'': the link https:// github.com/ XXX/ TODO should be updated.
\end{point}

\reply{Now updated. \revref
}



%%%%%%%%%%%%%%
\reviewersection{3}

\begin{quote}
The fundamental idea of this paper is important and the worked-out
algorithm appears quite valuable. I want to see it published. However
the current version is so difficult to read that it is likely to
have a very limited audience, and appears to have some errors.
\end{quote}

We're glad you think it's important, and apologies for the errors.

\begin{quote}
I do not feel I have understood the entire paper despite several
hours' work. Some of my comments may be misunderstandings, but if
so, the text I misunderstood could probably use clarification.
\end{quote}

As you'll see, we've tried to improve clarity throughout;
hopefully it is more digestable now.

\begin{point}{} % (1)
The description of \algorithmref{alg:extend} and \algorithmref{alg:mergeable} is confusing.
Naming them this way implies that they are separate algorithms but
in fact ``Algorithm 2'' is the definition of a function needed by
Algorithm 1.
\end{point}

\reply{
    We think it's reasonably standard for one algorithm to make use of another
    (for instance, many algorithms apply a \texttt{sort} at some point).
    However, it's a good point that referring abstractly to ``\algorithmref{alg:extend}'' and 
    ``\algorithmref{alg:mergeable}'' is a little abstract and hard to remember.
    So, we've gone back to places where the text just said ``\algorithmref{alg:extend}'',
    and where we thought helpful, changed it to something like
    ``extending haplotypes with \algorithmref{alg:extend}''.
    For more background:
    originally we had both \algorithmref{alg:extend} and \algorithmref{alg:mergeable} within 
	one algorithm environment.
    However, the \texttt{Mergeable} is a conceptually separate operation
    that by separating it out makes understanding \texttt{ExtendHaplotypes} easier.
	\algorithmref{alg:extend} is the first algorithm as it is the main construction and the most important of the two.
}

\begin{point}{} %(2)
\algorithmref{alg:mergeable} is mentioned in an explanation a full page before it
is introduced.
\end{point}

\reply{
    This is a consequence of the typesetting: \algorithmref{alg:mergeable} 
    is in a floating environment, and
    the earliest the environment will appear is the next page following the previous figure.
    Rest assured that Genetics will typeset this appropriately.
}

\begin{point}{} %(3)
\algorithmref{alg:extend} seems to be named ExtendHaplotypes and \algorithmref{alg:mergeable} seems
to be named Mergeable: having two names for each one is not helpful.
\end{point}

\reply{
    This point overlaps with point 3.1 -- see our response above for how we've hopefully
    improved the situation.
    We like to think of ExtendHaplotypes and Mergable as the names of the functions contained in \algorithmref{alg:extend} and \algorithmref{alg:mergeable}, respectively.
}

\begin{point}{} %(4)
``Mergeable'' is an unfortunate name for a function which does not
return a yes or no answer but instead returns the number of
new edges to be added. Also the text needs a comment about why
infinity is sometimes returned.
\end{point}

\reply{
    A reasonable point; we've changed the name to \texttt{MergeNum}.
	For clarity, we have also included a note about returning infinity in the description of \algorithmref{alg:mergeable}.
}

\begin{point}{} %(5)
In \algorithmref{alg:extend} the variable $m$ means the number of new edges to be
added. In \algorithmref{alg:mergeable} that quantity is called $m_e$ and $m$ means the
number of extended haplotypes. This is confusing. Please use the
same name for a given concept in all pseudocode.
\end{point}

\reply{Good point, this change has been made.
}

\begin{point}{} %(6)
Comments in the pseudocode would make it easier to follow, as could
use of more meaningful variable names (I particularly dislike $M$, $m$, $M'$
in the same algorithm).
\end{point}

\reply{Good point. We have included some comments for clarity.
However, we're going to respectfully differ on stylistic choice of concise variable names.
}

\begin{point}{}
Capital $I$ is an unfortunate subscript as it is almost indistinguishable
from digit 1.
\end{point}

\reply{While we agree in general,
within the context of our algorithm we are interested paths of inheritance 
	between the edges removed from a local tree (the out-forest) and the edges just added into the next local tree (the in-forest). 
	Because of our language use of out and in, we feel like our choice of $O$ and $I$ is reasonable.
}

\begin{quote} % (2)
\textbf{Issues in algorithms.}
\end{quote}

\begin{point}{}
\algorithmref{alg:extend} appears to have an error leading to an infinite
loop. Variable $M$ is initially set to 0. It is compared to variable $m$,
which is a number of new edges to be added and must be positive or zero.
Thus, $m$ will never be lower than $M$, so the action taken is to set $M'$ to
the minimum of $m$ and $M$, which will always be zero. $M$ is then set to
$M'$, that is, to zero. Since this is inside a while loop with terminating
condition $M$ == infinity, it does not terminate. (I coded it to check.)

The logic problem is probably around line 10. We have just tested
whether $m < M$ and this was false, so $m \ge M$. It is therefore
strange to take the minimum of $m$ and $M,$ as this will always be
$M$. However I can't work out the correct code.
\end{point}

\reply{
Good catch -- and, many apologies --
that is indeed a bug; the correction changes line 7 to
``if $m \le M$'' and line 10 to ``Set $M' = \min(m_e, M')$.''.
(To be clear, the bug was in the paper, not the implemented code;
if the reviewer is interested they can see our implementation
in the tskit library at
\href{https://github.com/tskit-dev/tskit/blob/c6b06b0c4f90b5098aedddca7dd0e3fbbad964d0/c/tskit/trees.c#L9078}{c/tskit/trees.c:9078} in the current release.)
}

\begin{point}{}
The text description of \algorithmref{alg:extend} (ExtendHaplotypes) says that
paths with the fewest new edges are added first. The pseudocode
given does not appear to do that. It appears (bearing in mind that
it has at least one bug, so I can't be positive) to be willing to
execute a Merge before examining all children, so I think it will
Merge the first child with a better score than previous children,
regardless of the fact that an even better child may remain to be
found.
\end{point}

\reply{
We apologize again for the bug in the previous point that caused this misunderstanding.
Fixed, it should perform as advertised
(and, we've added a bit clarifying why it works
to the description of \algorithmref{alg:extend}).
}

\begin{point}{}
After considerable work I have been unable to code \algorithmref{alg:mergeable}.
When is a node NULL? Does this mean the node doesn't exist (for
example, the parent of the root)?
\end{point}

\reply{
    A node is NULL when it has no parent; this is now explained in the description
    of the algorithm.
    Apologies, this is part of the conventions in tskit.
}

\begin{point}{} % (3)
I recommend that the example partial tree in \figref{fig:conceptual} be
extended to a full tree (it only needs one more node) and used in
the practical examples in the text.
\end{point}

\reply{
Good idea,
we now reference the figure in the text,
at \revref{} and \llname{ref_conceptual2}.
On further consideration we think that adding the rest of the tree
here would make the figure too confusing
(not so much for the trees in A\&C, but more so in the haplotype diagrams B\&D).
}

\begin{point}{} % (4) 
The concrete example under ``An algorithm to extend haplotypes'' is
difficult to visualize. Can it be related to \figref{fig:conceptual}? When I try to
do this, though, I have difficulty because in \figref{fig:conceptual} the right-hand
tree does not \emph{have} $v_1 \ldots v_n$, it jumps straight from $p$ to $c.$ Is
this example not mergeable, or is the description of mergeablility
missing any indication that some of the nodes it relies on may not
exist?
\end{point}

\reply{Good point. We have connected our concrete construction in that section
	 to reference \figref{fig:conceptual} to improve clarity. \revref
}

\begin{point}{} % (5) 
The direction in which time is measured on the tree needs to be stated
as unfortunately the literature has both conventions. Also, the text
implies at several points that this can be used for trees without
time information, but it cannot as times are used in the algorithm.
\end{point}

\reply{We state how time is measured here \revref;
we've added a reminder at \llname{time_reminder}.
We do use time in our algorithm, but mainly as a way to break ties, as is the case with \eqnref{eqn:matched-spans}. This tie break will default to the first node if the ARG does not contain time information.
We've also added a note about undated ARGs \llname{undated}.
}

\begin{point}{}
Following on from this, in \algorithmref{alg:extend}, why can no node in $u$ be at the
same time as a node in $v?$ At first I thought we were assuming that no
two nodes, other than samples, are ever at the same time (a fairly
common assumption) but then we would not need this rule. Is this just
a convenience to avoid putting two nodes on the same branch at the same
timepoint? What happens if this algorithm is attempted with inferred
ARGs that discretize time, as some inference algorithms do, so that ties
become common?
\end{point}

\reply{
    This is exactly right:
    we don't assume that node times are unique,
    as it's common (e.g., from discrete-time simulators)
    for this not to be true;
    so the check is to avoid producing an illegal situation
    (parent and child at the same time).
    We now explain this at \llname{undated}.
}

\begin{point}{\llname{truth}}
``In what follows we think of $T_2$ as the true ARG and $T_1$ as an
inferred ARG.'' This is not strongly enough stated. The measure is
asymmetrical with regards to $T_1$ and $T_2$, and will give wrong answers
if they are reversed. It is also not suitable for comparing two
ARGs where neither is the true ARG.
\end{point}

\reply{
    We've added extra emphasis to the point that the quantities are not symmetric
    \llname{asymmetric};
    furthermore the additional discussion in the ``Metrics on ARGs'' section
    should help reinforce this point.
    As for reversing $\T_1$ and $\T_2$,
	we think that while $\match(\T_2,\T_1)$ might not be the most useful quantity,
    it can still give some valuable information between inferred $\T_1$ and true $\T_2$.
    We discuss this when comparing inverse matched spans between $(\T_1,\T_2)$ and matched spans between $(\T_2, \T_1)$ \revref.
}

\begin{point}{}
In general the section on comparison of true and inferred ARGs is
difficult to follow. \figref{fig:conceptual_discrepancy} helps but appears too late in the
paper.
\end{point}

\reply{
    We have expanded our example to hopefully give more clarity (see additional text
    in caption).
    The first reference to \figref{fig:conceptual_discrepancy} occurs at the end
    of the second paragraph in the section on comparison;
    it would be hard to include the example earlier
    (and, we're not sure how helpful it will be until
    $\arf$ and $\tpr$ are defined).
}

\begin{point}{} % (7) 
In \figref{fig:conceptual_discrepancy} it is unfortunate to use to use an example where
the forward and inverse matches are the same. I would also appreciate
a brief explanation of why it's 46 and 47 in the denominator--I figured
it out but it's less than obvious. Why do sample nodes count, as they
cannot vary among trees? Doesn't this just add bookkeeping effort?
It's otherwise a very helpful figure though.
\end{point}

\reply{
    We sympathize with the reviewer that working through even a simple example like this
    is tricky and confusing (but, we also find it very helpful).
    Good point about the same forward and inverse matches.
    Note that we did construct this example so that matching for node 4 requires a max to be taken over shared spans of its two possible matches 4 and 9.
To show some variety, we have added computation of $\match(\T_2,\T_1)$ to show some additional variety and hopefully give more clarity to the methods themselves (see caption).
The samples are included because in practice they do \emph{not} necessarily always match:
consider a case where some of the samples are ancestors of others
(ancient samples; trios; etcetera).
}

\begin{point}{} % (8) 
I cannot clearly distinguish the three purplish colors in \figref{fig:speed_and_edges}.
\end{point}

\reply{Thank you for this feedback. We have changed \figref{fig:speed_and_edges} and the subsequent supplementary figures (\figref{fig:speed_and_edges_constant}, \figref{fig:speed_and_edges_other}, \figref{fig:speed_and_edges_other_constant}) to have more color variety.
}

\begin{point}{} % (9) 
The interpretation of \figref{fig:node-spans} is completely unclear to me. I
think it needs more explanatory text. I have no idea what point is being
made by these diagrams.
\end{point}

\reply{Thanks for the feedback. We have added substantial additional explanation at \revref{} and the caption of \figref{fig:node-spans}.
}

\begin{point}{} % (10) 
I remind the authors to replace the script availability placeholder
URL with the real thing! 
\end{point}

\reply{Noted and now changed, thank you!\revref
}

